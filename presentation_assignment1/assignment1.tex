\documentclass[shortpres]{beamer}
\usetheme{CambridgeUS}

% Depending on build configuration, one of these packages will
% enable unicode
%\usepackage[utf8]{inputenc}
\usepackage{fontspec}

\usepackage{caption}
\usepackage{subcaption}
\usepackage{multicol}
\usepackage{color}
\usepackage{psfrag} %for \psfrag in figures
\usepackage{pgfplots}
\usepackage{xmpmulti}

\usepackage{algorithm,algpseudocode}  %for algorithm environment

\setbeamertemplate{footline}
{
	\leavevmode%
	\hbox{%
		\begin{beamercolorbox}[wd=.333333\paperwidth,ht=2.25ex,dp=1ex,center]{author in head/foot}%
			\usebeamerfont{author in head/foot}\insertshortauthor%~~\beamer@ifempty{\insertshortinstitute}{}{(\insertshortinstitute)}
		\end{beamercolorbox}%
		\begin{beamercolorbox}[wd=.333333\paperwidth,ht=2.25ex,dp=1ex,center]{title in head/foot}%
			\usebeamerfont{title in head/foot}\insertshorttitle
		\end{beamercolorbox}%
		\begin{beamercolorbox}[wd=.333333\paperwidth,ht=2.25ex,dp=1ex,right]{date in head/foot}%
			\usebeamerfont{date in head/foot}\insertshortdate{}\hspace*{2em}
			\insertframenumber{} / \inserttotalframenumber\hspace*{2ex}
	\end{beamercolorbox}}%
	\vskip0pt%
}\part{title}
\beamertemplatenavigationsymbolsempty


%color specification---------------------------------------------------------------
\definecolor{TUMblue}{rgb}{0.00, 0.40, 0.74}
\definecolor{TUMgray}{rgb}{0.85, 0.85, 0.86}
\definecolor{TUMpantone285C}{rgb}{0.00, 0.45, 0.81}
\definecolor{lightblue}{rgb}{0.7529,0.8118,0.9333}

\setbeamercolor{block title}{fg=white, bg=TUMpantone285C}
\setbeamercolor{block body}{bg=lightblue}
\setbeamertemplate{blocks}[rounded][shadow=true]
%----------------------------------------------------------------------------------

\setbeamercolor{frametitle}{fg=TUMblue, bg=white}
\setbeamercolor{palette primary}{fg=TUMblue,bg=TUMgray}
\setbeamercolor{palette secondary}{use=palette primary,fg=TUMblue,bg=white}
\setbeamercolor{palette tertiary}{use=palette primary,fg=white, bg=TUMblue}
\setbeamercolor{palette quaternary}{use=palette primary,fg=white,bg=TUMpantone285C}


\setbeamercolor{title}{bg=white,fg=TUMblue}
\setbeamercolor{item projected}{use=item,fg=black,bg = lightblue}
\setbeamercolor{block title}{fg=black, bg=lightblue}
\setbeamercolor{block body}{bg=white}
\setbeamertemplate{blocks}[rounded][shadow=true]
%----------------------------------------------------------------------------------



%############### Self defined commands ##############################
\newcommand{\footer}[1]{\vfill \tiny{\textit{#1}}}
\newcommand{\footerstd}{\footer{Figures by Julius Ziegler, Philipp Bender, Thao Dang and Christoph Stiller:\\"Trajectory Planning for BERTHA - a Local, Continuous Method" in IEEE Intelligent Vehicles Symposium (IV), IEEE, 2014}}
%####################################################################

\usepackage{anyfontsize}

\title[{Tsunami simulation}]{Assignment 1}

\author[Bellamy Honal, Wieser]{George Bellamy, Christoph Honal, Felix Wieser\\\vspace{10pt}{\small Bachelorpraktikum}}
\institute[TU M\"unchen]{Technische Universit\"at M\"unchen}

\date{7. November 2017}

\begin{document}
\maketitle

\begin{frame}{Shock-Shock Problem}
	%TODO: Insert content here
\end{frame}

\begin{frame}{Rare-Rare Problem}
	%TODO: Insert content here
\end{frame}

\begin{frame}{Dam-Break}
	%TODO: Insert content here
\end{frame}

%TODO: Remove / comment the following slides, when finished
\begin{frame}{Template: image centered, text below}
	\begin{figure}[t]
	\includegraphics[clip, width=.6\linewidth]{img/dummy_image.jpg}
	\caption*{Dummy image}
	Die Crash Test Dummies sind eine um 1840 gegründete, weltweit operierende Gang von Verkehrssündern. Zahlreiche, teils spektakuläre, Unfälle gehen auf das Konto dieser von Interpol, CIA und FDP gesuchten Organisation (stupidedia.org)
	\end{figure}
\end{frame}

\begin{frame}{Template: content left, content right}
	\begin{multicols}{2}
		\begin{figure}[t]
			\includegraphics[clip, width=0.98\linewidth]{img/dummy_image.jpg}
			\caption*{Dummy image}
		\end{figure}		
		
	\columnbreak
	
		Die Crash Test Dummies sind eine um 1840 gegründete, weltweit operierende Gang von Verkehrssündern. Zahlreiche, teils spektakuläre, Unfälle gehen auf das Konto dieser von Interpol, CIA und FDP gesuchten Organisation (stupidedia.org)
	\end{multicols}
\end{frame}

\begin{frame}{Mathegedöns}
	\begin{multicols}{2}
		
		Aufzählung
		\begin{itemize}
			\item Das ist
			\item eine wirklich
			\item sinnvolle Aufzählung
		\end{itemize}
		
		\vspace{10pt}
		
		Nummerierung
		\begin{enumerate}
			\item Eins
			\item Zwei
			\item Drölf
		\end{enumerate}
		
		\vspace{10pt}
		
		\begin{tabular}{ll}
			Name & Leistung\\
			\hline
			Heinz & Nicht so toll\\
			Anton & Besser\\
			Julian & Alles gewonnen (Streber)
		\end{tabular}
		
	\columnbreak
		
		Matrix \hspace{10pt}
		$
			\begin{bmatrix}
				a & b & c\\
				d & e & f\\
				g & \dots & 42
			\end{bmatrix}
			\cdot
			\begin{pmatrix}
			a & b & c\\
			d & e & f\\
			g & \dots & 42
			\end{pmatrix}
			\cdots
		$
		
	\end{multicols}
	
\end{frame}
\end{document}